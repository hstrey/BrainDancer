\documentclass{article}

% Language setting
% Replace `english' with e.g. `spanish' to change the document language
\usepackage[english]{babel}

% Set page size and margins
% Replace `letterpaper' with `a4paper' for UK/EU standard size
\usepackage[letterpaper,top=2cm,bottom=2cm,left=3cm,right=3cm,marginparwidth=1.75cm]{geometry}

% Useful packages
\usepackage{amsmath}
\usepackage{graphicx}
\usepackage[colorlinks=true, allcolors=blue]{hyperref}

\title{Rotated Ellipse}
\author{Helmut H. Strey}

\begin{document}
\maketitle

\begin{abstract}
Rotated Ellipses
\end{abstract}

\section{Introduction}

Here we collect calculations that pertain to the intensity fitting functions for rotated Ellipses.

\section{Theory}

\subsection{Circle}
Equation of the circle
\begin{equation}
    x^{2}+y^{2}=R^{2}
\end{equation}
Intensity distribution of a circle with width $\sigma$
\begin{equation}
    I(x,y)=A\exp{-(R-\sqrt{x^{2}+y^{2}})^{2}/\sigma^{2}}
\end{equation}
\begin{figure}
\centering
\includegraphics[width=0.3\textwidth]{circle.png}
\caption{\label{fig:frog}heatmap of intensity distribution of circle of radius 1}
\end{figure}

\subsection{Rotated Ellipse}
Equation for rotated Ellipse:
\begin{equation}
    \dfrac {((x-x_{0})\cos(\theta)+(y-y_{0})\sin(\theta))^2}{(R_x)^2}+\dfrac{((x-x_{0}) \sin(\theta)-(y-y_{0}) \cos(\theta))^2}{(R_y)^2}=1
\end{equation}
This equation can be parameterized using the angle $\alpha$ in the following way:
\begin{equation}
    \begin{aligned}
            x(\alpha) &= R_x \cos(\alpha) \cos(\theta) - R_y \sin(\alpha) \sin(\theta) + x_{0} \\
y(\alpha) &= R_x \cos(\alpha) \sin(\theta) + R_y \sin(\alpha) \cos(\theta) + y_{0}
    \end{aligned}
\end{equation}
Using a similar approach to the section above we can write:
\begin{equation}
    I(x,y)=A\exp\left({-(1-\sqrt{\dfrac {((x-x_{0})\cos(\theta)+(y-y_{0})\sin(\theta))^2}{(R_x)^2}+\dfrac{((x-x_{0}) \sin(\theta)-(y-y_{0}) \cos(\theta))^2}{(R_y)^2}})^{2}/\sigma^{2}}\right)
\end{equation}
\begin{figure}
\centering
\includegraphics[width=0.3\textwidth]{ellipse.png}
\caption{\label{fig:ellipse}heatmap of intensity distribution of ellipse of $r_{x}=1$, $r_{y}=2$ and $\theta=\pi/3$}
\end{figure}
\subsection{Angled cylinder}
Here we discuss that there are two potential reasons why we see an ellipse.  The first reason is that if the cylinder is not perfectly
aligned with the z-axis then we see the projection of a circle on a plane that is inclined by an angle $\alpha$.  There is also the possibility that the image is deformed by image distortion that is related to the MRI imaging process (edge effects).

Let us first consider the first possiblity since it is mathematically rigorous.  To define the rotation axis we need four parameters: the point $(x_{0},y_{0})$ at $z=0$ and the two angles $(\alpha,\vartheta)$ that describe how the line crosses, where $\alpha$ is the angle of the line projected onto the $(x,y)$-plane, and $\vartheta$ is the angle of the line with respect to the z-axis.  $\vartheta$ is the angle of the cylinder with respect to the z-axis and a cylinder with radius $R$ is deformed into an ellipse with axis $R$ and $R/cos(\vartheta)$.

Such a line can be written mathematically as:
\begin{equation}
(x(z),y(z)) = (z\cos(\vartheta)\cos(\alpha)+x_{0},z\cos(\vartheta)\sin(\alpha)+y_{0})
\end{equation}
\end{document}